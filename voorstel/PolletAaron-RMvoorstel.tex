%==============================================================================
% Sjabloon onderzoeksvoorstel bachproef
%==============================================================================
% Gebaseerd op document class `hogent-article'
% zie <https://github.com/HoGentTIN/latex-hogent-article>

% Voor een voorstel in het Engels: voeg de documentclass-optie [english] toe.
% Let op: kan enkel na toestemming van de bachelorproefcoördinator!
\documentclass{hogent-article}

% Invoegen bibliografiebestand
\addbibresource{voorstel.bib}
% Informatie over de opleiding, het vak en soort opdracht
\studyprogramme{Professionele bachelor toegepaste informatica}
\course{Research Methods}
\assignmenttype{Onderzoeksvoorstel}
% Voor een voorstel in het Engels, haal de volgende 3 regels uit commentaar
% \studyprogramme{Bachelor of applied information technology}
% \course{Bachelor thesis}
% \assignmenttype{Research proposal}

\academicyear{2022-2023} % TODO: pas het academiejaar aan

% TODO: Werktitel
\title{beoordelen van de risico's van het gebruik van AI-gestuurde ontwikkeling zoals 
GitHub Co-pilot in het softwareontwikkelingsproces: zorgen voor veiligheid en betrouwbaarheid}

% TODO: Studentnaam en emailadres invullen
\author{Aaron Pollet}
\email{aaron.pollet@student.hogent.be}

% TODO: Geef de co-promotor op
\supervisor[Co-promotor]
{S. Beekman (Synalco, \href{mailto:sigrid.beekman@synalco.be}{sigrid.beekman@synalco.be})}

% Binnen welke specialisatierichting uit 3TI situeert dit onderzoek zich?
% Kies uit deze lijst:
%
% - Mobile \& Enterprise development
% - AI \& Data Engineering
% - Functional \& Business Analysis
% - System \& Network Administrator
% - Mainframe Expert
% - Als het onderzoek niet past binnen een van deze domeinen specifieer je deze
%   zelf
%
\specialisation{Mobile \& Enterprise development}
\keywords{ChatGPT, World Wide Web, Artificial Intelligence, Software Development, Github Co-pilot}

\begin{document}

\begin{abstract}
{Deze paper onderzoekt de risico’s van AI-geassisteerde software ontwikkeling. 
Dit omvat het gebruik van AI-assistenten zoals Github Copilot en bevat een beoordeling op vlak van 
veiligheid en betrouwbaarheid. Deze studie introduceert de context en bekijkt daarnaast de toenemende acceptatie 
van AI-technologieën binnen het softwareontwikkelingsproces. Het doel van het onderzoek is om voor 
softwareontwikkelaars de risico’s in kaart te brengen en aanbevelingen te doen om de risico’s te beperken. 
De voorgestelde methodologie omvat een uitgebreide literatuurstudie waar de context van het onderzoek wordt bepaald. 
Uit voorlopige resultaten blijkt dat hoewel AI- gestuurde softwareontwikkeling nieuwe risico’s introduceert, 
deze vermeden kunnen worden mits de juiste aanpak. De waarde van dit onderzoek ligt in de duidelijke schetsing van de 
mogelijke valkuilen en daartegenover concrete aanbevelingen voor ontwikkelaars die AI gebruiken om de veiligheid en 
betrouwbaarheid van hun code te waarborgen. Algemeen wil dit onderzoek bijdragen aan het groeiend gebruik van 
AI-gestuurde ontwikkeling en de invloed ervan op veiligheid en betrouwbaarheid van de software.}
\end{abstract}

% \tableofcontents

% % De hoofdtekst van het voorstel zit in een apart bestand, zodat het makkelijk
% % kan opgenomen worden in de bijlagen van de bachelorproef zelf.
% \input{voorstel-inhoud}

% \printbibliography[heading=bibintoc]

% \end{document}